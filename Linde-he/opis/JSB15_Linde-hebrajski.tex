\documentclass[12]{mwart}
\usepackage{polyglossia}
\setdefaultlanguage{polish}

\usepackage{enumitem}
\usepackage{draftwatermark}

\usepackage{xltxtra}

\setmainfont[Mapping=tex-text]{TeX Gyre Termes}
\setsansfont[Mapping=tex-text]{TeX Gyre Adventor}
%\setmainfont{TeXGyreTermes}
%\setmainfont{DejaVu Serif}
%\setmainfont{Bitstream Vera Serif}
%\setmonofont{TeX Gyre Cursor}
\setmonofont{DejaVu Sans Mono}
\usepackage{draftwatermark}

% \usepackage{bibentry,natbib}

\usepackage{graphicx}

\usepackage{hyperref}

\usepackage{soul}

\usepackage{relsize}

% \usepackage[style=authoryear,natbib=true]{biblatex}
% %\addbibresource{JSB2013.bib,typografia.bib}
% \addbibresource{4JSB2014.bib}
% \AtEveryBibitem{\clearfield{note}}



\newcommand{\program}[1]{\textsf{#1}}

\title{Język hebrajski w słowniku Lindego\\Analiza przypadku}
\author{Janusz S. Bień}

\date{16.03.2015, 13.04.2018}

% \date{24.03.2014, 8.04.2014, ??.10.2014}

\begin{document}
\maketitle
% \pagestyle{empty}

% no math
\catcode`\&=12
\catcode`\_=12

\begin{quote}
  Tekst na otwartej licencji Creative Commons Uznanie Autorstwa,
  źródła dostępne w repozytorium
  \url{https://bitbucket.org/jsbien/ilindecsv}.
\end{quote}

\section{Wstęp}
\label{sec:wstp}

W pliku \path{1h.csv} znajduje się przykładowy indeks do korekty
wyrazów pisanych alfabetem hebrajskim występujących w pierwszym tomie
słownika. Może służyć do testowania narzędzi.

W pliku \path{2h.csv} znajduje się przykładowy indeks do korekty
wyrazów pisanych alfabetem hebrajskim występujących w drugim tomie
słownika.

Indeks można w zasadzie weryfikować i poprawiać za pomocą
\texttt{djview4poliqarp}.

Można wykorzystywać rownież odpowiedni edytor tekstowy, ale wymaga to
nabrania wprawy w edytowaniu tekstów o mieszanym kierunku pisma.

Planowany tryb korekty w programie \texttt{djview4poliqarp}
powinien być najlepsza metodą.

Indeksy tego typu powinne być tworzone automatycznie. Opisana dalej
procedura utworzenia tego indeksu ręcznie stanowi punkt wyjścia do
sformułowania odpowiednio szczegółowego algorytmu.

\section{Wyszukanie słów w alfabecie hebrajskim}
\label{sec:wyszukanie-sow-w}

Wyszukanie słów było kłopotliwe, ponieważ wyszukiwarka działa obecnie
tylko na starych skanach. Punktem wyjścia była kwerenda \texttt{Syr},
trafienia za pomocą programu \texttt{djview4poliqarp} zostały zapisane
w formie pliku \texttt{csv}. Z pliku tego usunięto niepotrzebne
kolumny w celu nadania mu formy indeksu. Po otworzeniu indeksu w
\texttt{djview4poliqarp} wszystkie hasła zostały przejrzane. Hasła
odnoszące się do przytoczeń w alfabecie hebrajskim zostały
zmodyfikowane:
\begin{itemize}
\item dopasowane zaznaczenia do wyrazu w alfabecie hebrajskim,
\item hasło zostało zastapione numerem strony.
\end{itemize}
Dodatkowo utworzono hasła dla innych wyrazów w alfabecie hebrajskim
znajdującym się w sąsiedztwie.

W trakcie pracy ujawniła się wada programu \texttt{djview4poliqarp}
polegajaca na tym, że dla wyświetlonego hasła nie ma metody
wyświetlenia odpowiedniej strony za pomocą \texttt{djview}, co
pozwoliłoby zlokalizować stronę w strukturze słownika za pomocą
,,konspektu'' (w ,,starych skanach'' cały słownik stanowił formalnie
jeden dokument).. W konsekwencji numery tomów zostały ustalone przez
posortowanie w Emacsie wynikowego pliku według URL (z niejasnych
powodów dało to kilka błędów, które zostały poprawione ręcznie).

Na podstawie numeru tomu i numeru strony możliwe było zlokalizowanie
szukanych słów w nowych skanach (wygodnie było to robić równolegle z
oglądaniem trafień na starych skanach).

Niestety w ten sposób udało się znależć tylko niewielką część słów
pisanych tym alfabetem.

\section{Lokalizowanie i wycinanie słów w alfabecie hebrajskim}
\label{sec:lokal-i-wycin}

Początkowo program \texttt{djview4poliqarp} nie pozwalał indeksować
dowolnego dokumentu DjVu. Najpierw został stworzony ,,ślepy'' indeks
zawierający w charakterze hasła tylko numer tomu i numer strony (w
razie potrzeby powtórzone odpowiednią liczbę razy), a potem za pomocą
\texttt{djview} były tworzone URL dopisywane do pliku. Okazało się
jednak to niewygodne, dla drugiego tomu utworzono ślepy indeks
zawierający bezpośredni kontekst w komentarzu.
% Już tego nie pamiętam, a obecnie tego opisu nie rozumiem :-(

W ogólnym wypadku wycinki powinne być robione automatycznie na
podstawie URL, i pobierane od razu z binarnej maski. 

Obecnie konieczne było stosowanie funkcji programu \texttt{djview}
zapisywania zaznaczenia w pliku graficznym --- stosowany był format
\texttt{png} (bez konkretnego powodu). Niestety wydaje się, że tak
utworzone pliki mają rozdzielczość zależną od jakichś przygodnych
czynników, nie stanowią więc rzeczywistych wycinków oryginału.

\section{OCR słów w języku hebrajskim}
\label{sec:ocr-sow-w}

Wybrałem najprostszą, choć może nie najlepszą drogę --- uzyskane
wycinki wykorzystałem jako dane do FineReadera. Wiekszość liter
FineReader --- jak się wydaje --- rozpoznał poprawnie. Kilka razy
niewątpliwie się pomylił. Kilka razy źle okreslił gabaryty słów, po
ich ręcznej modyfikacji rozpoznał słowa lepiej (a w każdym razie
inaczej). Niektórych słów nie rozpoznał w ogóle - były to słowa
nadmiernie powiększone przy robieniu wycinków.

Wyniki zostały zapisane w formacie PDF i w formie czystego tekstu. PDF
został skonwertowany do DjVu, aby mieć jednocześnie dostęp do skanu i
tekstu. Wydaje się jednak, że \textit{pdf2djvu} zmienił kolejność
znaków hebrajskich.

Dla tomu drugiego zapisano wyniki w formie czystego tekstu i tam
kolejność znaków była prawidłowa.

\section{Nanoszenie wyników OCR na indeks}
\label{sec:nanosz-wynik-ocr}

Zadanie to powinno być oczywiście wykonywane automatycznie, robienie
tego ręcznie okazało się tak niewdzięczne, że próbka została
ograniczona tylko do pierwszych dwóch tomów.

Istotnym problemem jest kierunek pisma:
\begin{itemize}
\item na jakim etapie przetwarzania zmieniła się kolejność liter ---
  sprawa ta wymaga wyjaśnienia,
\item na litery hebrajskie Emacs reaguje przejsciem w tryb pisania od
  prawe do lewej, co jest mocno dezorientujace i sprzyja pomyłkom.
\end{itemize}

Kształt znaków hebrajskich w domyślnym foncie (DejavuSans?) różni się
znacznie od ich kształtów w słowniku Lindego. Wskazane jest używanie
fontu bardziej przypominajacego oryginalny (Cardo?, Linux
Libertine?). Jednak obecnie indeks w \texttt{djview4poliqarp}
wyświetlany jest za pomocą fontu systemowego.

\section{Zakończenie}
\label{sec:zakoczenie}

Ręczne przygotowanie indeksu okazało się trudniejsze, niż się
wydawało.

\end{document}

%%% Local Variables:
%%% mode: latex
%%% TeX-master: t
%%% TeX-engine: xetex
%%% TeX-PDF-mode: t
%%% coding: utf-8
%%% End:
