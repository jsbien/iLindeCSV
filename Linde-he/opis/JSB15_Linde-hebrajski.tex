\documentclass[12]{mwart}
\usepackage{polyglossia}
\setdefaultlanguage{polish}

\usepackage{enumitem}
\usepackage{draftwatermark}

\usepackage{xltxtra}

\setmainfont[Mapping=tex-text]{TeX Gyre Termes}
\setsansfont[Mapping=tex-text]{TeX Gyre Adventor}
%\setmainfont{TeXGyreTermes}
%\setmainfont{DejaVu Serif}
%\setmainfont{Bitstream Vera Serif}
%\setmonofont{TeX Gyre Cursor}
\setmonofont{DejaVu Sans Mono}
\usepackage{draftwatermark}

% \usepackage{bibentry,natbib}

\usepackage{graphicx}

\usepackage{hyperref}

\usepackage{soul}

\usepackage{relsize}

% \usepackage[style=authoryear,natbib=true]{biblatex}
% %\addbibresource{JSB2013.bib,typografia.bib}
% \addbibresource{4JSB2014.bib}
% \AtEveryBibitem{\clearfield{note}}



\newcommand{\program}[1]{\textsf{#1}}

\title{Język hebrajski w słowniku Lindego\\Analiza przypadku}
\author{Janusz S. Bień}

\date{16.03.2015, 10.10.2018}

% \date{24.03.2014, 8.04.2014, ??.10.2014}

\begin{document}
\maketitle
% \pagestyle{empty}

% no math
\catcode`\&=12
\catcode`\_=12

\begin{quote}
  Tekst na otwartej licencji Creative Commons Uznanie Autorstwa,
  źródła dostępne w repozytorium
  \url{https://bitbucket.org/jsbien/ilindecsv}.

  \bigskip
  Indeksy zostały przygotowane w 2015 i w tym samym czasie opisane
  nieco zbyt skrótowo w niniejszym tekście. W 2018 r. dodano
  uzupełnienia i objaśnienia w formie przypisów lub uwag w inny sposób
  wyróżnionych typograficznie.
\end{quote}

\section{Wstęp}
\label{sec:wstp}

W pliku \path{1h.csv} znajduje się przykładowy indeks do korekty
wyrazów pisanych alfabetem hebrajskim występujących w pierwszym tomie
słownika. Może służyć do testowania narzędzi.\footnote{Planowane
  narzędzie nie powstały.}

W pliku \path{2h.csv} znajduje się przykładowy indeks do korekty
wyrazów pisanych alfabetem hebrajskim występujących w pierwszym tomie
słownika. 

\begin{quote}
  \textit{Udostępnione są również robocze indeksy, w tym również dla
    innych tomów.}[2018]
\end{quote}

Indeks można w zasadzie weryfikować i poprawiać za pomocą
\texttt{djview4poliqarp}\footnote{Patrz
  np. \url{https://www.slideshare.net/jsbien/jsb-i-linde181001ipi-117452985}}.

Można wykorzystywać również odpowiedni edytor tekstowy, ale wymaga to
nabrania wprawy w edytowaniu tekstów o mieszanym kierunku pisma.

Planowany tryb korekty w programie \texttt{djview4poliqarp} powinien
być najlepsza metodą\footnote{Częściowo nieaktualny tekst \textit{Tryb korekty w
    djview4poliqarp} z 2015 r. jest obecnie dostępny w repozytorium
  \url{https://bitbucket.org/jsbien/linde-info}. Proponowane tam
  rozszerzenia programu \textsf{djview4poliqarp} nie wydają się
  obecnie tak bardzo istotne.}.

Indeksy tego typu powinny być tworzone automatycznie. Opisana dalej
procedura utworzenia tego indeksu ręcznie stanowi punkt wyjścia do
sformułowania odpowiednio szczegółowego algorytmu\footnote{W aktualnej
  wersji \textsf{djview4poliqarp} tworzenie indeksów jest nieco
  łatwiejsze, swoją drogą zmianie ulegly też priorytety.}.

\section{Wyszukanie słów w alfabecie hebrajskim}
\label{sec:wyszukanie-sow-w}

Wyszukanie słów było kłopotliwe, ponieważ wyszukiwarka działa obecnie
tylko na starych skanach\footnote{Nieaktualne. Stare skany to inaczej
  wersja 2010, patrz
  \url{https://szukajwslownikach.uw.edu.pl/slownik-lindego/}.}. Punktem
wyjścia była kwerenda \texttt{Syr}, trafienia za pomocą programu
\texttt{djview4poliqarp} zostały zapisane w formie pliku
\texttt{csv}. Z pliku tego usunięto niepotrzebne kolumny w celu
nadania mu formy indeksu. Po otworzeniu indeksu w
\texttt{djview4poliqarp} wszystkie hasła zostały przejrzane. Hasła
odnoszące się do przytoczeń w alfabecie hebrajskim zostały
zmodyfikowane:
\begin{itemize}
\item dopasowane zaznaczenia do wyrazu w alfabecie hebrajskim,
\item hasło zostało zastąpione numerem strony.
\end{itemize}
Dodatkowo utworzono hasła dla innych wyrazów w alfabecie hebrajskim
znajdującym się w sąsiedztwie.

W trakcie pracy ujawniła się wada programu \texttt{djview4poliqarp}
polegająca na tym, że dla wyświetlonego hasła nie ma metody
wyświetlenia odpowiedniej strony za pomocą \texttt{djview}, co
pozwoliłoby zlokalizować stronę w strukturze słownika za pomocą
,,konspektu'' (w ,,starych skanach'' cały słownik stanowił formalnie
jeden dokument)\footnote{Wada ta została usunięta dopiero 2 2018
  r. (wersja 3 programu).}. W konsekwencji numery tomów zostały
ustalone przez posortowanie w Emacsie wynikowego pliku według URL (z
niejasnych powodów dało to kilka błędów, które zostały poprawione
ręcznie).

Na podstawie numeru tomu i numeru strony możliwe było zlokalizowanie
szukanych słów w nowych skanach\footnote{Chodzi o skany w wyższej
  rozdzielczości i w skali szarości, a nie czarno białe, jak wersja
  2010. Jest to tzw. wersja 2016, patrz
  \url{https://szukajwslownikach.uw.edu.pl/slownik-lindego-nowy/}.}
(wygodnie było to robić równolegle z oglądaniem trafień na starych
skanach).

Niestety w ten sposób udało się znaleźć tylko niewielką część słów
pisanych tym alfabetem.\footnote{Stwierdzenie niejasne, do
  zweryfikowania przez powtórzenie wyszukiwania bezpośrednio na nowych
  skanach.}.

\section{Lokalizowanie i wycinanie słów w alfabecie hebrajskim}
\label{sec:lokal-i-wycin}

Początkowo program \texttt{djview4poliqarp} nie pozwalał indeksować
dowolnego dokumentu DjVu\footnote{Niejasne jest użycie czasu
  przeszłego, do ewentualnej weryfikacji w historii
  programu.}. Najpierw został stworzony ,,ślepy'' indeks zawierający w
charakterze hasła tylko numer tomu i numer strony (w razie potrzeby
powtórzone odpowiednią liczbę razy), a potem za pomocą \texttt{djview}
były tworzone URL dopisywane do pliku. Okazało się jednak to
niewygodne, dla drugiego tomu utworzono ślepy indeks zawierający
bezpośredni kontekst w komentarzu.
% Już tego nie pamiętam, a obecnie tego opisu nie rozumiem :-(

W ogólnym wypadku wycinki powinny być robione automatycznie na
podstawie URL\footnote{\label{przypis}Okazuje się, że było to od dawna
  możliwe przy pomocy mało znanej opcji programu \textsf{ddjvu} ---
  patrz \url{https://sourceforge.net/p/djvu/feature-requests/95/}.}, i
pobierane od razu z binarnej maski\footnote{Ma to sens tylko wtedy,
  gdy mamy pewność, że binaryzacja jest poprawna}.

Obecnie konieczne\footnote{Okazalo się to nieprawdą --- patrz przypis
  nr \ref{przypis}.} było stosowanie funkcji programu \texttt{djview}
zapisywania zaznaczenia w pliku graficznym --- stosowany był format
\texttt{png} (bez konkretnego powodu)\footnote{Funkcję taką ma również
  \textsf{djview4poliqarpq}, widocznie nie był stosowany ze względu na
  problem wspomniany wcześniej.}. Niestety wydaje się, że tak
utworzone pliki mają rozdzielczość zależną od jakichś przygodnych
czynników, nie stanowią więc rzeczywistych wycinków oryginału.

\section{OCR słów w języku hebrajskim}
\label{sec:ocr-sow-w}

Wybrałem najprostszą, choć może nie najlepszą drogę --- uzyskane
wycinki wykorzystałem jako dane do FineReadera. Większość liter
FineReader --- jak się wydaje --- rozpoznał poprawnie. Kilka razy
niewątpliwie się pomylił. Kilka razy źle określił gabaryty słów, po
ich ręcznej modyfikacji rozpoznał słowa lepiej (a w każdym razie
inaczej). Niektórych słów nie rozpoznał w ogóle - były to słowa
nadmiernie powiększone przy robieniu wycinków.

Wyniki zostały zapisane w formacie PDF i w formie czystego tekstu. PDF
został skonwertowany do DjVu, aby mieć jednocześnie dostęp do skanu i
tekstu. Wydaje się jednak, że \textit{pdf2djvu} zmienił kolejność
znaków hebrajskich.

Dla tomu drugiego zapisano wyniki w formie czystego tekstu i tam
kolejność znaków była prawidłowa.

\begin{quote}
  Obecnie można rozważyć wykorzystanie Tesseracta. [2018]
\end{quote}

\section{Nanoszenie wyników OCR na indeks}
\label{sec:nanosz-wynik-ocr}

Zadanie to powinno być oczywiście wykonywane automatycznie, robienie
tego ręcznie okazało się tak niewdzięczne, że próbka została
ograniczona tylko do pierwszych dwóch tomów\footnote{Dla pozostałych
  tomów dostępne są tylko indeksy z lokalizacją, ale bez treści
  haseł.}.

Istotnym problemem jest kierunek pisma:
\begin{itemize}
\item na jakim etapie przetwarzania zmieniła się kolejność liter ---
  sprawa ta wymaga wyjaśnienia,
\item na litery hebrajskie Emacs reaguje przejściem w tryb pisania od
  prawe do lewej, co jest mocno dezorientujące i sprzyja pomyłkom.
\end{itemize}

Kształt znaków hebrajskich w domyślnym foncie (DejavuSans?) różni się
znacznie od ich kształtów w słowniku Lindego. Wskazane jest używanie
fontu bardziej przypominającego oryginalny (Cardo?, Linux
Libertine?). Jednak obecnie indeks w \texttt{djview4poliqarp}
wyświetlany jest za pomocą fontu systemowego.

\section{Zakończenie}
\label{sec:zakoczenie}

Ręczne przygotowanie indeksu okazało się trudniejsze, niż się
wydawało.

\begin{quote}
  Trudność brała się z nieznajomości alfabetu hebrajskiego i
  konieczności wykorzystywania automatycznego rozpoznawania znaków,
  oraz z braku wprawy w edycji tekstów o dwóch kierunkach
  pisma. Całkowicie ręczne wpisywanie haseł do indeksu w programie
  \textsf{djview4poliqarp} byłoby chyba znacznie łatwiejsze.
\end{quote}

\end{document}

%%% Local Variables: 
%%% ispell-local-dictionary: "polish"
%%% eval: (flyspell-mode 1)
%%% coding: utf-8-unix
%%% mode: latex
%%% TeX-master: t
%%% TeX-PDF-mode: t
%%% TeX-engine: xetex
%%% End: 
