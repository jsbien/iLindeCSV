% C-x 8 RET (translated from C-x 8 <return>) runs the command
% ucs-insert, which is an interactive compiled Lisp function in
% `mule-cmds.el'.
\documentclass{beamer}
\usepackage{fontspec}
\usepackage{polyglossia}
\setmainlanguage{polish}
\usepackage{xltxtra}

\setmainfont[Mapping=tex-text]{TeX Gyre Termes}
\setsansfont[Mapping=tex-text]{TeX Gyre Termes}
%\setmainfont{TeXGyreTermes}
%\setmainfont{DejaVu Serif}
%\setmainfont{Bitstream Vera Serif}
\setmonofont{TeX Gyre Cursor}
%\setmonofont{DejaVu Sans Mono}

\usepackage{relsize}
%\usepackage{color}
\usepackage{graphicx}
\usepackage{hyperref}

\newcommand{\y}[1]{\colorbox{yellow}{#1}}
\newcommand{\g}[1]{\colorbox{green}{#1}}


\setbeamercovered{highly dynamic}

\setbeamertemplate{navigation symbols}{} %no nav symbols

% to much space for content list?:
%\usetheme{Warsaw}
%\usetheme{Copenhagen}
% not bad:
%\usetheme{Montpellier}
%\usetheme{Antibes}
\usetheme{JuanLesPins}

\usepackage{natbib}

\usepackage{graphicx}

\usepackage{url}

\usepackage{hyperref}

%\font\linkfont="DejaVu Sans Mono"



    % 2015-02-18
    % 2015-02-25
    % 2015-03-04
    % 2015-03-11
    % 2015-03-18
    % 2015-03-25
    % 2015-04-01
    % 2015-04-08
    % 2015-04-15
    % 2015-04-22

    % 2015-04-29
    % 2015-05-06
    % 2015-05-13
    % 2015-05-20
    % 2015-05-27
    % 2015-06-03 

\title[Dygitalizacja polskich słowników
historycznych (3322-DPSH-OG) 2015-02-25 (2/16)]{\textsc{Konwersatorium}\\Dygitalizacja polskich
  słowników historycznych}

\author{prof. dr hab. Janusz S. Bień}

\institute{Katedra Lingwistyki Formalnej\\Wydział Neofilologii\\Uniwersytet Warszawski}

\date[2014]{3322-DPSH-OG\\(przedmiot ogólnouniwersytecki humanistyczny)}

%\url{jsbien@uw.edu.pl}

\begin{document}

\begin{frame}
  \frametitle{Nazwy własne w słowniku Lindego}
  \url{http://wbl.klf.uw.edu.pl/76/}

  \begin{quote}
    Wojciech Ryszard Rzepka, Bogdan Walczak\\
Nazwy własne w słowniku języka polskiego\\
Samuela Bogumiła Lindego\\
\url{http://wbl.klf.uw.edu.pl/176/}
  \end{quote}
 \begin{itemize}
  \item Dostępny indeks haseł pisanych dużą literą (Moodle)\\
    \url{https://bitbucket.org/jsbien/ilindecsv/}
  \item Należy 
    \begin{itemize}
    \item w razie potrzeby odnaleźć hasło w słowniku
    \item sprawdzić, czy hasło jest nazwą własną,
    \item jeśli tak, to jakiego typu,
    \item wpisać odpowiedni komentarz
    \end{itemize}
  \item 1 zadanie to 100 haseł (można więcej)
  \end{itemize}
\end{frame}


\begin{frame}
  \frametitle{Typy nazw własnych}

  Klasyfikacja w zasadzie według artykułu Rzepki i Walczaka:
  \begin{enumerate}
  \item nazwy miejscowości (\alert{t}oponimy), np. \textit{Adryanopol},
    \alert{\textit{Adiga}};
  \item nazwy części świata, państw i ich części oraz krain
    geograficznych (\alert{c}horonimy), np. \textit{Algier};
  \item nazwiska, przezwiska i imiona (\alert{a}ntroponimy),
    np. \textit{Aaron}, \textit{Achacy};
  \item \alert{e}tnonimy [nazwa ludu, rodu, klanu, plemienia, szczepu,
    narodu], np. \textit{Afer};
    \item nazwy \alert{h}erbowe, np. \textit{Abdank}
  \end{enumerate}

\end{frame}


\begin{frame}
  \frametitle{Typy nazw własnych}
  Uzupełnienia:
  \begin{itemize}
  \item \alert{błąd} (np. Abelek),\\
{\relsize{-2}\url{http://wbl.klf.uw.edu.pl/19/1/iiLinde.djvu?djvuopts=&page=235&zoom=width&showposition=0.5,0.62&highlight=1606,863,139,52}}
\item \alert{p}ochodne, np. \textit{Aaronowy}
\item \alert{i}nne
  \end{itemize}
Uszczegółowienia:
\begin{itemize}
\item toponim: miasto, wieś, rzeka itp.
\item \ldots
\end{itemize}
\end{frame}

\begin{frame}[fragile]
  \frametitle{Przykłady}
\relsize{-1}
\begin{verbatim}
  Adryanopol    ::264-4-33-0-0-0 __ _ ___:: t:miasto
  Algier    ::296-4-37-0-0-0 __ _ ___::  c:królestwo
  Aaron    ::273-1-44-0-0-0 __ _ ___:: a:kapłan
  Afer    ::296-3-41-0-0-0 __ _ ___:: e:Afrykanin
  Abdank    ::254-3-30-0-0-0 __ _ ___:: h:
  Abelek;    ::235-4-37-0-0-0 __ _ ___:: błąd 
  Aaronowy;    ::370-2-13-0-0-0 __ _ ___:: p:Aaron
\end{verbatim}
\texttt{kapłan} ? \texttt{Afrykanin} ? może pominąć?

\alert{i} --- na razie brak przykładu
\end{frame}

\end{document}



%%% Local Variables: 
%%% coding: utf-8-unix
%%% mode: latex
%%% TeX-master: t
%%% TeX-PDF-mode: t
%%% TeX-engine: xetex 
%%% End: 
